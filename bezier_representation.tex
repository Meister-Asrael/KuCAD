\chapter{Bézier Representation}
Recall that the set of polynomials of degree n forms a vector space. A
basis is \[\{1, x, \dots, x^n\}\].
\begin{defi} A representation ofa polyomial with respect
  to \[\{1, x, \dots, x^n\}\] is colled the monomial representation.
\end{defi}
\section{Bernstein Polynomial}

\[1 = 1^n
    = (u + (1-u))^n 
    = \sum^n_{i=0} {n \choose i} u^i (1-u)^{n-i}\]
\begin{defi}
\[B^n_i(u) = {n \choose i}u^i (1-u)^{n-i}\] is called a Bernstein
Polynomial.
\end{defi}
\(B_i^n (u) = 0\) if \((i < 0)\) or \((i > n)\).
\begin{lem}
\label{lem2.1}
\(B_0^n \dots B_n^n\) are linear independent.
\end{lem}
\begin{proof}
\begin{eqnarray*}
  \sum b_i B_i^n &=& \sum b_i {n \choose i} u^i (1-u)^{n-1} \\
                &=& \frac{0}{\frac{1}{(1-u)^n}}\\
                &\Rightarrow& \sum b_i (\frac{u}{1-u})^i = 0\\
                &\Rightarrow& \sum b_i s^i = 0\\
                &\Rightarrow& b_i = 0
\end{eqnarray*}
\end{proof}
\begin{theo}
\(B_0^n \dots B_n^n \) form a basis for polynomials of degree n. Proof
follows from Lemma \ref{lem2.1} and the fact that the dimension of the
space of polynomials of degree \(n\) is \(n+1\)
\end{theo}
\begin{lem}{Symmetry}
\[B_i^n(0) = B^n_{n-1}(u)\]
\end{lem}
\begin{lem}
\begin{equation*}
  B_i^n(0) = B^n_{n-1}(1) = 
  \begin{cases}
    1 & \textrm{if } i=0 \cr
    0 & \textrm{otherwise} \cr
  \end{cases} 
\end{equation*}
\end{lem}
\begin{lem}
\(B_0^n \dots B_0^n \) form a partition of unity (\(\sum B_i^n = 1\))
\end{lem}
\begin{lem}
\(B_i^n(u)>0\), \(u\in(0,1)\)
\end{lem}
\begin{lem}
  \(B_i^{n+1}(u) = u B_{i-1}^n(u) + (1-u) B_i^n(u) \)
\end{lem}
\section{Bézier Representation}
\begin{defi}
A representation of a polynomial with respect to \( {B_0^n \dots B_n^n}
\) is calles the Bézier representation. Let \( c(u) = \sum_{i=0}^n c_i
B_i^n(u)\). \(c_i\) can be in 
\( \mathbb{R}^n \)
. For practical reasons
is \(u \in [a, b]\).\\
Note: \(u(t)=a(1-t)+bT \).\\
\[ b(t) := c(a(t))\], b has same the degree as c and represents the
same polinomial, but with a different parametrisation.
\end{defi}

\begin{figure}[h]
\centering
\includegraphics[width=.2\textwidth]{graphics/kurve.pdf}
\end{figure}

\(b(t)\) has a Bézierrepresentation of degree n.
\[ b(t) = \sum_{i=0}^n b_iB_i^n(t) \nonumber
\]
\begin{defi} \(b_i\) is called a control point.\end{defi}
\begin{defi} u is a global parameter, t is local.\end{defi}
\begin{defi} The piecewise linear interpolant of the \(b_i\) is called the
  control polygone.

\begin{figure}[h]
\centering
\includegraphics[width=.25\textwidth]{graphics/control_polygone.pdf}
\caption{control polygone}
\end{figure}
\end{defi}
\begin{lem}

\begin{eqnarray*}
b(u) &=& \sum_{i=0}^n b_i B_i^n(t)\\
     &=& \sum_{i=0}^n b_{n-i}B_{n-i}^n(1-t)
\end{eqnarray*}
\end{lem}
\begin{lem}[end point interpolation]
\[ \mathbb{b}(a)=\mathbb{b}_0\]
\end{lem}
