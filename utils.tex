%extended symbol support
\usepackage{marvosym}					
\usepackage{amssymb}
\usepackage{amsmath}
\usepackage{cancel}
\usepackage{centernot}
\usepackage{pgf,tikz}
\usepackage{pgfplots}
\usepackage{xfrac}
\usepackage{mathtools}
\usepackage{mdframed}
\usepackage{hyperref}
\usepackage{enumerate}

\usetikzlibrary{arrows}
\usetikzlibrary{patterns}


%common TikZ packages
\usepackage{tikz}
\usetikzlibrary{matrix}

%use this if you want to write something that not affects the position of the following text
\newcommand{\zwidth}[1]{\makebox[0pt]{#1}}

%use this for clasped underbraces
\newcommand{\underbracec}[6]{\left#5\vphantom{#3 #1 #4}\right. #3 \underbrace{#1}_{#2} #4 \left.\vphantom{#3 #1 #4}\right#6}

% provides a fast method to create vectors
%\newcommand{\vecz}[2]{\begin{pmatrix}#1\\#2\end{pmatrix}}
%\newcommand{\vect}[3]{\begin{pmatrix}#1\\#2\\#3\end{pmatrix}}
\newcommand{\vecz}[2]{\left[#1\:\:#2\right]^{\text{t}}}

%Use only in Math mode:
% makes the root of #1#2#3. Underbraces #2 with #4.
\newcommand{\uroot}[4]
{
	\sqrt
	{
		\vphantom{#1#2#3} \smash
		{
     		#1\underbrace{#2}_{#4}#3
     	}
    }
	\vphantom{#1\underbrace{#2}_{#4}#3}
}

%from Guy Rutenberg http://www.guyrutenberg.com/2008/01/04/add-explanations-to-latex-formulas/
\newcommand{\explain}[2]{\underset{\mathclap{\overset{\uparrow}{#2}}}{#1}}
\newcommand{\explainup}[2]{\overset{\mathclap{\underset{\downarrow}{#2}}}{#1}}

%\widebar wider bar! from http://www.latex-community.org/forum/viewtopic.php?f=46&t=20717
\newcommand*\widebar[1]{\hbox{\vbox{\hrule height 0.5pt
      \kern0.5ex\hbox{\kern-0.1em\ensuremath{#1}\kern-0.1em}}}}
      
%Multiline comments
%\newcommand{\comment}[1]{}

%Starts enumaration of sections with 0 in a new part. Copy this in your mastersheet. Keep it commented here because you cannot turn it off.
%\let\randocommandname\part
%\renewcommand{\part}[1]{\randocommandname{#1}\setcounter{section}{0}}

%Usefull commands and operators
\newcommand{\N}{\mathbb{N}}		%Natürliche Zahlen
\newcommand{\R}{\mathbb{R}}		%Reelle Zahlen
\newcommand{\Z}{\mathbb{Z}}		%Ganze Zahlen
\newcommand{\Q}{\mathbb{Q}}		%Rationale Zahlen
\newcommand{\K}{\mathbb{K}}		%allgemeiner Körper
\newcommand{\C}{\mathbb{C}}		%Komplexe Zahlen
\newcommand{\dV}{V^{\star}}		%Dualraum von V
\DeclareMathOperator{\dist}{d}	%Abstand
\DeclareMathOperator{\HW}{HW}	%Häufungswert
\DeclareMathOperator{\KM}{KM}	%???
\DeclareMathOperator{\HP}{HP}	%Häufungspunkt
\DeclareMathOperator{\grad}{grad}	%Grad
\DeclareMathOperator{\arsinh}{arsinh}	
\DeclareMathOperator{\arcosh}{arcosh}
\DeclareMathOperator{\artanh}{artanh}
\DeclareMathOperator{\id}{id}		%Identität
\DeclareMathOperator{\Kern}{Kern}	%Kern
\DeclareMathOperator{\Bild}{Bild}	%Bild
\DeclareMathOperator{\Hom}{Hom}		%Menge der Homomorphismen
\DeclareMathOperator{\CP}{CP}	%Charakteristisches Polynom
\DeclareMathOperator{\MP}{MP}	%Minimalpolynom
\DeclareMathOperator{\ER}{ER}	%Eigenraum
\DeclareMathOperator{\EW}{EW}	%Eigenwert
\DeclareMathOperator{\GL}{GL}	%Menge der invertierbaren Matrizen
\DeclareMathOperator{\Abb}{Abb}	%Abbildung
\DeclareMathOperator{\Rang}{Rang}	%Rang
\DeclareMathOperator{\Spec}{Spec}	%Spektrum
\DeclareMathOperator{\Spur}{Spur}	%Spektrum
\DeclareMathOperator{\Haupt}{Hauptraum}
\newcommand{\infsumo}{\sum\limits_{n=1}^{\infty}}
\newcommand{\infsumz}{\sum\limits_{n=0}^{\infty}}
\newcommand{\limi}{\lim\limits_{x\rightarrow\infty}}
\newcommand{\limn}{\lim\limits_{x\rightarrow 0}}
\newcommand{\limp}{\lim\limits_{x\rightarrow 0+}}
\newcommand{\limm}{\lim\limits_{x\rightarrow 0-}}
\newcommand{\eqlhop}{\underset{\text{de l'Hôpital}}{=}}
\newcommand{\convinf}{\underset{n\rightarrow\infty}{\longrightarrow}}
\newcommand{\fann}{\forall n\in\N:}
\newcommand{\deff}{\underset{\text{Def}}{\Rightarrow}}
\newcommand{\limninf}{\lim\limits_{n\rightarrow\infty}}
\newcommand{\satz}[1]{\textbf{Satz:} #1\vspace{1cm}}
\newcommand{\beispiel}[1]{ \textbf{Beispiel:} #1\vspace{.5cm}}
\newcommand{\define}[1]{\textbf{Definition:} #1\vspace{.5cm}}

\newtheorem{defi}{Definition}[chapter]
\newtheorem{lem}{Lemma}[chapter]
\newtheorem{theo}{Theorem}[chapter]
